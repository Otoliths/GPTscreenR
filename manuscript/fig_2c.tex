\documentclass{article}
\usepackage[T1]{fontenc}
\usepackage{xparse}
\usepackage{enumitem}
\setlist[description]{
  font={\sffamily\bfseries},
  labelsep=0pt,
  labelwidth=\transcriptlen,
  leftmargin=\transcriptlen,
}

\newlength{\transcriptlen}

\NewDocumentCommand {\setspeaker} { mo } {%
  \IfNoValueTF{#2}
  {\expandafter\newcommand\csname#1\endcsname{\item[#1:]}}%
  {\expandafter\newcommand\csname#1\endcsname{\item[#2:]}}%
  \IfNoValueTF{#2}
  {\settowidth{\transcriptlen}{#1}}%
  {\settowidth{\transcriptlen}{#2}}%
}

% Easiest to put the longest name last...
\setspeaker{GPT}[GPT-4]
\setspeaker{User}
\setspeaker{System}

% How much of a gap between speakers and text?
\addtolength{\transcriptlen}{1em}%

\usepackage[paperheight=100cm]{geometry}

\begin{document}
\pagestyle{empty}
\begin{description}

  \System You are being used to help researchers perform a scoping review. You
  are not interacting directly with a user.

  A scoping review is a type of systematic review used to map the published
  scholarship on a topic. To gather relevant sources for a scoping review, the
  researchers search bibliographic databases for sources that match a selected
  Population, Concept, and Context (the inclusion criteria). The titles and
  abstracts of sources that are found in this search search are then screened
  against the inclusion criteria.

  Your task is to screen a single source against the inclusion criteria. You
  will be provided with the review objective and inclusion criteria, and then
  you will then be provided with the source title and abstract. You will then
  be instructed to work step by step through the process of comparing the
  source against the inclusion criteria. Finally, you will instructed to make a
  recommendation on whether the source should be included.

  The next message will be from the user, and will contain the scoping review
  objective and inclusion criteria.

  \User OBJECTIVE: This scoping review will examine the existing primary
  research on the role of therapy alpacas in enhancing the mental health and
  emotional well-being of elderly residential aged care facility residents
  
  POPULATION: Elderly people living in residential aged care facilities.
  `Elderly' is defined as 65 years of age or older. 

  CONTEXT: Residential aged care facilities that have used therapy alpacas as
  part of their programme of care for residents. `Residential aged care
  facility' is defined as a residential setting intended as a long-term place
  of residence for elderly people which includes provision of support for
  activities of daily living (e.g. meal preparation, bathing, housekeeping) and
  nursing support (e.g. medication management). Such facilities will also
  typically offer other structured programmes and facilities to provide
  entertainment, diversion, and wellbeing. It excludes other residential
  settings intended for elderly people that do no provide daily living or
  nursing supports (e.g. independent living villages) or that are not long-term
  (e.g. hospitals or hospices). 

  CONCEPT: The impact of therapy alpaca programmes on stress reduction,
  emotional well-being, mental health, overall life satisfaction, or similar
  outcomes for residents.

  \System Let's work step by step. First, generate a numbered list of
  statements that summarise the inclusion criteria for the scoping review,
  including the Population, Concept, and Context. The statements should be
  clear, comprehensive and complete. Any source for which all the statements
  are true is a source that meets the inclusion criteria. As a template, here
  are some example statements (these are a generic set of examples that are not
  related to the current scoping review):

    \begin{enumerate}[label=\arabic*.\ ]
    \item The source reports the results of a randomised control trial
    \item The source reports the results of a study in which:
      \begin{enumerate}[label=\arabic{enumi}\alph*.\ ]
      \item The participants were all male; AND
      \item The participants were all aged between 18 and 74 inclusive
      \end{enumerate}
    \item The source reports the results of a study conducted in the European
          Union.
    \end{enumerate}

    Aspects of the inclusion criteria with multiple elements should be broken
    down into separate points where possible. For example, instead of:

    \begin{enumerate}[label=\arabic*.\ ]
      \item The source reports on a study of men who live in the European
            Union.
    \end{enumerate}

    You should instead say:

    \begin{enumerate}[label=\arabic*.\ ]
    \item The source reports on a study of people who are:
      \begin{enumerate}[label=\arabic{enumi}\alph*.\ ]
        \item Male; and
        \item Living in the European Union.
      \end{enumerate}
    \end{enumerate}

  \GPT \begin{enumerate}[label=\arabic*.\ ]
    \item The source includes primary research conducted on elderly people 1a.
    `Elderly' is defined as 65 years or older. 
    \item The elderly people in the source reside in a residential aged care
    facilities 
      \begin{enumerate}[label=\arabic{enumi}\alph*.\ ]
        \item `Residential aged care facility' is a long-term residence for
        elderly people providing support for activities of daily living and
        nursing support. 
        \item The facility offers structured programmes and facilities for
        entertainment, diversion, and well-being. 
        \item The facility does not include residential settings that do not
        provide daily living support or nursing support, or that are not for
        long-term residence. 
      \end{enumerate}
    \item The residential aged care facilities mentioned in the source have
    implemented therapy alpaca programs as part of their care for residents. 
    \item The source examines the impact of therapy alpaca programmes 
      \begin{enumerate}[label=\arabic{enumi}\alph*.\ ]
        \item The impact examined includes effects on stress reduction,
        emotional well-being, mental health, overall life satisfaction, or
        similar outcomes for residents.
      \end{enumerate}
    \end{enumerate}

  \System The next message will be from the user, and will contain the title
  and abstract of a source to be compared against the inclusion criteria.

  \User TITLE: Therapy Ungulate Visits Reduce Depression, Anxiety, and Stress
  in Australian Nursing Home Residents

  ABSTRACT: Nursing home residents typically have rates of depression, anxiety,
  and stress (DAS) significantly higher than those of equivalent populations
  living independently. As well as reducing quality of life and increasing
  morbidity and mortality, DAS significantly increases the burden on both
  nursing homes and health systems in caring for the elderly. Recent pilot
  trials of the use of therapy ungulates (alpacas, llamas, and camels) have
  shown promising initial results in reducing DAS among nursing home residents.
  We conducted a randomised control trial involving 8,124 residents across 17
  nursing homes in urban and rural Australia in which residents were randomised
  to either a weekly 2-hour therapy camel visit or equivalent time in a
  television viewing lounge. Residents receiving the intervention had
  significantly (p < 0.01) reduced levels of depression (mean DASS-21 score for
  depression reduction 8.2 points), anxiety (mean reduction 11 points) and
  stress (mean reduction 13 points) compared to the control arm. This study
  represents the first large-scale randomised trial of ungulate therapy for
  nursing home residents, and provides strong support for its effectiveness in
  reducing DAS.

  \System Let's continue to work step by step. Refer back to the set of
  statements you developed summarising the inclusion criteria. For each
  statement, decide whether or not the statement is true for the source
  described by the title and abstract. You must select from the following
  permitted responses: TRUE, FALSE, LIKELY TRUE, LIKELY FALSE, or NOT
  APPLICABLE. No other response is permitted. It is normal for the title and
  abstract to not have enough information to make a clear decision for every
  statement. There is a natural and normal amount of ambiguity in this process.
  For these situations, give your best guess, making use of your general
  knowledge, and deciding LIKELY TRUE or LIKELY FALSE. Responses like UNCLEAR
  or NOT ENOUGH INFORMATION are not permitted. After giving your response, give
  a one sentence explanation for your response.

  \GPT 1. The source includes primary research conducted on elderly people. -
  LIKELY TRUE: While the age of the nursing home residents is not explicitly
  stated, it can be reasonably inferred that they are likely to be elderly. -
  Explanation: `Nursing home residents' are generally senior individuals.

  1a. `Elderly' is defined as 65 years or older. - LIKELY TRUE: Although the
  abstract does not explicitly define an age for ``elderly'', it is common
  understanding that elderly often refers to individuals who are 65 years or
  older. - Explanation: Although not explicit, the elderly population in most
  research is usually referred to those aged 65 or more.

  2. The elderly people in the source reside in a residential aged care
  facilities. - TRUE: The residents are shown to live in nursing homes, which
  qualifies as a type of residential aged care facility. - Explanation: Nursing
  homes are a type of residential aged care facility.

  2a. `Residential aged care facility' is a long-term residence for elderly
  people providing support for activities of daily living and nursing support.
  - LIKELY TRUE: Although not specifically mentioned in the abstract, it is
  understood that nursing homes facilitate activities of daily living and
  nursing support for the elderly. - Explanation: It is widely known that
  nursing homes provide support for activities of daily living and nursing
  care.

  2b. The facility offers structured programmes and facilities for
  entertainment, diversion, and well-being. - TRUE: The abstract mentions pilot
  trials of therapy ungulates, suggesting the presence of structured programmes
  at the facility. - Explanation: The implementation of therapy ungulate trials
  indicates the existence of structured programs for the residents' well-being.

  2c. The facility does not include residential settings that do not provide
  daily living support or nursing support, or that are not for long-term
  residence. - LIKELY TRUE: Although not explicitly stated, nursing homes
  generally provide both daily living support and nursing care, and are meant
  for long-term residence of the elderly. - Explanation: As part of their
  operations, nursing homes generally provide these forms of supports and cater
  to long-term residents.

  3. The residential aged care facilities mentioned in the source have
  implemented therapy alpaca programmes as part of their care for residents. -
  TRUE: The abstract mentions the use of therapy ungulates, which includes
  alpacas, as part of their program. - Explanation: The abstract identifies
  therapy ungulates, including alpacas, being utilized in the nursing home.

  4. The source examines the impact of therapy alpaca programmes. - FALSE:
  Although the abstract mentions the use of therapy ungulates, it specifically
  mentions the use of therapy camels in the trial, not alpacas. - Explanation:
  The abstract does not mention the specific effects of therapy alpaca
  programs, but rather, the therapy camel program.

  4a. The impact examined includes effects on stress reduction, emotional
  well-being, mental health, overall life satisfaction, or similar outcomes for
  residents. - TRUE: The study indeed examines the impact on stress,
  depression, and anxiety in the nursing home residents. - Explanation: The
  abstract mentions that the study measures impact on stress reduction,
  depression, and anxiety, which can be seen as elements of mental health and
  emotional well-being.

  \System Let's continue to work step by step. Consider your decisions on
  whether the title and abstract meet the conclusion criteria. Overall, is it
  likely true that the source meets the inclusion criteria? Reply with a single
  word, either INCLUDE or EXCLUDE, representing your recommendation on whether
  the source is likely to meet the inclusion criteria. You must reply with a
  single word only and it must be one of these two words; any other reply will
  cause the automatic parsing of your response to fail, which will be
  troublesome for the user.

  \GPT EXCLUDE

\end{description}
\end{document}
