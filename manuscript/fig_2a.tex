% https://chat.openai.com/share/95f22bfc-1f00-439f-947f-22b6e7ae1ffa

\documentclass{article}
\usepackage[T1]{fontenc}
\usepackage{xparse}
\usepackage{enumitem}
\setlist[description]{
  font={\sffamily\bfseries},
  labelsep=0pt,
  labelwidth=\transcriptlen,
  leftmargin=\transcriptlen,
}

\newlength{\transcriptlen}

\NewDocumentCommand {\setspeaker} { mo } {%
  \IfNoValueTF{#2}
  {\expandafter\newcommand\csname#1\endcsname{\item[#1:]}}%
  {\expandafter\newcommand\csname#1\endcsname{\item[#2:]}}%
  \IfNoValueTF{#2}
  {\settowidth{\transcriptlen}{#1}}%
  {\settowidth{\transcriptlen}{#2}}%
}

% Easiest to put the longest name last...
\setspeaker{GPT}[GPT-4]
\setspeaker{User}
\setspeaker{System}

% How much of a gap between speakers and text?
\addtolength{\transcriptlen}{1em}%

\usepackage[paperheight=100cm]{geometry}

\begin{document}
\pagestyle{empty}
\begin{description}

  \User You are being used to help researchers perform a scoping review. A
  scoping review is a type of systematic review used to map the published
  scholarship on a topic. To gather relevant sources for a scoping review, the
  researchers search bibliographic databases for sources that match a selected
  Population, Concept, and Context (the inclusion criteria). The titles and
  abstracts of sources that are found in this search search are then screened
  against the inclusion criteria.

  This review has the following objective and inclusion criteria:

  OBJECTIVE: This scoping review will examine the existing primary research on
  the role of therapy alpacas in enhancing the mental health and emotional
  well-being of elderly residential aged care facility residents

  POPULATION: Elderly people living in residential aged care facilities.
  `Elderly' is defined as 65 years of age or older.

  CONTEXT: Residential aged care facilities that have used therapy alpacas as
  part of their programme of care for residents. `Residential aged care
  facility' is defined as a residential setting intended as a long-term place
  of residence for elderly people which includes provision of support for
  activities of daily living (e.g. meal preparation, bathing, housekeeping) and
  nursing support (e.g. medication management). Such facilities will also
  typically offer other structured programmes and facilities to provide
  entertainment, diversion, and wellbeing. It excludes other residential
  settings intended for elderly people that do no provide daily living or
  nursing supports (e.g. independent living villages) or that are not long-term
  (e.g. hospitals or hospices).

  CONCEPT: The impact of therapy alpaca programmes on stress reduction,
  emotional well-being, mental health, overall life satisfaction, or similar
  outcomes for residents.

  Here is the title and abstract of the source for you to screen:

  TITLE: Therapy Ungulate Visits Reduce Depression, Anxiety, and Stress in
  Australian Nursing Home Residents

  ABSTRACT: Nursing home residents typically have rates of depression, anxiety,
  and stress (DAS) significantly higher than those of equivalent populations
  living independently. As well as reducing quality of life and increasing
  morbidity and mortality, DAS significantly increases the burden on both
  nursing homes and health systems in caring for the elderly. Recent pilot
  trials of the use of therapy ungulates (alpacas, llamas, and camels) have
  shown promising initial results in reducing DAS among nursing home residents.
  We conducted a randomised control trial involving 8,124 residents across 17
  nursing homes in urban and rural Australia in which residents were randomised
  to either a weekly 2-hour therapy camel visit or equivalent time in a
  television viewing lounge. Residents receiving the intervention had
  significantly (p < 0.01) reduced levels of depression (mean DASS-21 score for
  depression reduction 8.2 points), anxiety (mean reduction 11 points) and
  stress (mean reduction 13 points) compared to the control arm. This study
  represents the first large-scale randomised trial of ungulate therapy for
  nursing home residents, and provides strong support for its effectiveness in
  reducing DAS.

  Please respond with a single word, either INCLUDE or EXCLUDE, representing
  your recommendation 

  \GPT INCLUDE

\end{description}
\end{document}
